\documentclass[letterpaper,11pt,colorlinks=true,allcolors=blue]{article}

%% Language and font encodings
\usepackage[english]{babel}
\usepackage[utf8x]{inputenc}
\usepackage[T1]{fontenc}

%%\usepackage{hyperref}
%%\hypersetup{
%%    colorlinks=true,
%%    linkcolor=blue,
%%    filecolor=magenta,      
%%    urlcolor=blue,
%%}

%% Sets page size and margins
\usepackage[letterpaper,top=3cm,bottom=2cm,left=3cm,right=3cm,marginparwidth=1.75cm]{geometry}

%% Useful packages
\usepackage{amsmath}
\usepackage{amssymb}
\usepackage{graphicx}
\usepackage[colorinlistoftodos]{todonotes}
\usepackage[colorlinks=true, allcolors=blue]{hyperref}
\usepackage{colortbl}

\title{NBUA Statements of Policies \\ and Procedures}
%% \author{You}

\begin{document}
\begin{figure}
\centering
\includegraphics[width=0.3\textwidth]{NBUALogo.jpg}
\end{figure}


\maketitle

The Northwest Baseball Umpires Association (NBUA), a Washington corporation, hereinafter referred to as NBUA, is a non-profit organization formed for amateur baseball umpires that provides services to teams and leagues who need umpires to officiate their games.  NBUA is a member organization governed by a Board of Directors elected by a majority vote of the members.

NBUA shall provide all members with a variety of services including training, support, the opportunity to advance their skills and the opportunity to officiate games with other well-trained umpires. NBUA shall also facilitate assigning of umpires, billing of umpire fees and payment of those fees earned by umpires. The following policies and procedures have been developed for the membership of NBUA to provide a guide for activities and expectations within the NBUA. These policies and procedures are intended to provide a reasonable and common-sense approach to continuing and further enhancing this organization and to ensure all members understand the requirements and expectations of association membership.  This document will be subject to periodic review and revision by the Board of Directors.

The following are policies and procedures developed by the association to provide expectations to the membership. These policies and procedures are subject to interpretation and modification by the Board of Directors.

\newpage
\section{GENERAL MEMBERSHIP}
NBUA facilitates training, provides umpire services to leagues, tournaments, and teams (including assigning of games), collects fees from clients, and disburses fees to umpires.

\begin{itemize}
\item Membership: To become and remain a “member in good standing” each year the following must occur:
\begin{itemize}
\item Payment of annual association dues (for continuing members, unless otherwise directed the Association will deduct dues from a members earnings for the next year’s dues).
\item Attending the minimum training required for tier placement.
\item Executing the \href{http://www.nbua.net/members/msa/}{Member Services Agreement (MSA)}.
\item Adherence to the Bylaws of the NBUA as posted on the NBUA web site.
\item Compliance with the Association Policies and Procedures.
\item Be of good moral character.
\item Completing a membership application (first year umpires only).
\end{itemize}

\item Independent contractor: All association members shall provide umpiring services as an independent contractor and as such shall be responsible for the following: 
\begin{itemize}
\item Providing, at their cost, all uniforms and equipment necessary to provide umpiring services. Uniforms and equipment shall meet NBUA standards and NBUA requirements.
\item NBUA shall report to the Internal Revenue Service (IRS) all earnings and payments made to association members and will provide member an IRS Form 1099. Payment of all taxes shall be the responsibility of the member acting as an independent contractor.
\end{itemize}

\item Game assignments: Section six (6) of the policies and procedures addresses specific details and information required to receive assignments. It should be noted as stated in article 2.2 of the MSA, membership in NBUA does not guarantee any game assignment whatsoever. The following is an overview of association policies regarding assignments.
\begin{itemize}
\item Assignments shall be made by the association Assignor based on a variety of factors including but not limited to:
\begin{itemize}
\item Members published availability.
\item Members published geographical preference. 
\item Matching members tier level to the tier requirement of each game.
\item Association assigning policy.
\end{itemize}
\end{itemize}

\item First Year Umpire: First year umpires wishing to join NBUA must do the following:
\begin{itemize}
\item Complete all of the “Membership” criteria listed above.
\item Pay additional training fees as required.
\item Complete training requirement as prescribed by the training group.
\item Successfully complete training to the satisfaction of the Director of Training and his/her staff.
\item Satisfy the requirements to become a \href{https://woa.arbitersports.com/front/104759/Site}{Washington Officials Association (WOA)} registered official, or show proof of having passed a background check.
\end{itemize}
Upon satisfactory completion of the above, first year umpires shall be placed at the “Apprentice” tier. 

\item Transferring Membership: Experienced umpires who move to our area before the beginning of a season will be considered “first year” umpires and will complete the same membership and training requirements as any “first year” umpire. The transferring member may provide a letter of recommendation from his/her previous association's President or Assigning Secretary that includes the number of games umpired annually, the level of games umpired and any other relevant information. If this information is provided, at the discretion of the Board of Directors, the following may occur:

\begin{itemize}
\item Umpire may be allowed to attend higher tier training.
\item At the beginning of the season, transferring umpire may be assigned one or more games at a level of the association’s choice where the umpire will work with a senior umpire and will be evaluated.
\item Based on this evaluation the association will determine the level of games appropriate and the Board of Directors may consider assigning the umpire to a different tier level.
\end{itemize}

\item Mid-Season Membership: To accommodate an experienced umpire who moves to our area after the season has started, at the discretion of the Board of Directors the following may occur:
\begin{itemize}
\item Umpire must complete membership requirements listed above with the obvious exception of attending training.
\item Umpire will be assigned one or more games at a level of the association’s choice where the umpire will work with a senior umpire and will be evaluated. 
\item Based on this evaluation the association will determine if the umpire’s training and skill levels meet our minimum requirement and if they do, the level of games that will be assigned.
\item If it is determined that the umpire’s training and skill level do not meet NBUA expectations, no games will be assigned that year and the umpire will be invited to return the next year and apply as a “first year” umpire. 
\end{itemize}
\end{itemize}

\newpage
\section{NBUA EXPECTATIONS}
As a membership association the following expectations must be followed to ensure NBUA continues to enjoy the reputation of being a quality organization providing highly professional umpires. 

\subsection*{Standard Uniforms:} 
The following uniform requirement shall apply to all NBUA games:
\begin{itemize}
\item Jersey: Black with white trim (piping), Undershirt shall be black or white providing partners must match. Numbers or association patches are not allowed.
\item Slacks: Heather gray or charcoal gray. It is preferred but not mandatory that slacks match among crew members. 
\item Belt: Black, high gloss or plain leather finish. While not required it is preferred that the belt be a 1 ¾ inch as provided by umpire equipment dealers. Under no circumstances shall a large belt buckle or a belt buckle with graphics or words be used.
\item Socks: Black
\item Shoes: Black with some white allowed for manufacturer logos or decoration.
\item Hat: Black fitted. Hat shall have NBUA stitched on front in black block letters with white outline. NBUA has contracted with “official” association hat suppliers, while it is not required to purchase hats from these suppliers; any hat purchased from another vendor must match the hats provided from association authorized vendors. 
\item Ball bags: Black
\item Helmets: If a helmet is worn by the plate umpire it shall be black and shall not have drawings or other graphics other than the logo of the manufacturer.
\end{itemize}

\subsection*{Optional Uniforms:} 
The following are alternative uniform combinations for NBUA games:
\begin{itemize}
\item Jersey: Any colored jersey or undershirt is optional, provided partners are matching.
\item Long sleeve jersey: Long sleeve versions of jerseys are authorized.
\item Jacket: Any recognized jacket is acceptable.
\item Plate Coat: Plate coat shall always be an option for plate umpire, but if worn on bases undershirt shall be black and crew must match.
\end{itemize}

Note: if a jacket is worn by one field umpire, then all field umpires must wear jackets, however it is not required that the plate umpire wear a jacket. In this event the color of the plate umpire's jersey must match with the field umpire coat color. 

Hats bearing the NBUA logo shall not be used in any game not scheduled through NBUA and shall not be worn outside NBUA functions.

\newpage
\begin{flushleft}\textbf{Use of alcohol or drugs:}\end{flushleft}
Using substances the day of the game that could impair judgment or other cognitive or psycho-motor skills (i.e., alcohol, marijuana, or even certain prescription drugs) prior to officiating a game is strictly forbidden. If an umpire arrives for an NBUA game and his/her partner detects any evidence of substances referred to above, or by the actions of the umpire, it is suspected, an emergency call must be made immediately to the Assignor.  If the assignor is not reached, contact must be made immediately to the President or another member of the Board of Directors. Under no circumstances shall the umpire suspected of alcohol or drug use be permitted to enter the playing field. 

Arriving at a game under the influence of any such substance is grounds for immediate suspension of membership in the NBUA.

\begin{flushleft}\textbf{Other impairments:}\end{flushleft}
If an umpire arrives at a game and is suspected or observed to be impaired for any other reason including but not limited to physical injury, use of prescription drugs, etc., to a degree where he/she cannot effectively umpire the game or where umpiring the game could be dangerous to the umpire or any participant an emergency call must be made immediately to the Assignor.  If the assignor is not reached, contact must be made immediately to the President or another member of the Board of Directors. Under no circumstances shall the umpire suspected of being impaired be permitted to enter the playing field. 

\begin{flushleft}\textbf{Parking lot expectations:}\end{flushleft}
Rarely are locker rooms available for NBUA umpires thus we must normally prepare for and conclude our games in a parking lot. While not an ideal situation, the following is expected:
\begin{itemize}
\item  Parking location: Umpires shall attempt to park in a secure yet private location and whenever possible in a location where they cannot be openly observed changing into their uniform.
\item  Uniform changing decorum: In all circumstances the umpire must exercise good judgment when changing clothes in a parking lot. Specifically, when on school grounds care must be taken to avoid students.
\item  Tobacco products prohibition: Smoking or the use of any tobacco like products, including “Vapes”, “ECigs”, or the like, are prohibited at all times at any school location being serviced by NBUA. This applies both during school contests as well as during non-school contests using any school facility.  This includes all parking areas within the school facility or any other location that restricts tobacco use.
\item  Use of alcohol after a game: Use of alcohol after completing a game and while still in the presence or potentially in the view of teams, parents or others is strictly prohibited. Consumption of alcohol is prohibited at all times at any location being serviced by NBUA.  This applies both during school contests as well as during non-school contests.  This includes all parking areas in or adjacent to the facility.
\item  Discussions with parents, players, coaches: While good judgment should be exercised, we discourage NBUA umpires to engage in discussions with parents, players or coaches in the parking lot after the game. 
\end{itemize}
In all cases NBUA members are expected to exercise good judgment and not enter into activities that could compromise the integrity of the association. 

\section{TRAINING}
Umpire training is provided each year to help umpires improve their skills and learn from the experiences of others. Training classes are established by tier level to provide appropriate training; all umpires regardless of their tier or experience level are required to attend training. 

\begin{itemize}
\item All umpires are required to attend a minimum number of hours of training as specified by their tier placement.
\item If an umpire cannot attend a specific training class and wants to substitute a class at a different tier level prior approval from the Director of Training is required.
\item If an umpire is not able to attend a minimum of training due to their work schedule, attending school out of the Seattle area or being out of town during training sessions they must contact the Director of Training to make suitable arrangements.
\item Training prepares umpires and ensures quality officiating for the upcoming season; thus training must be completed prior to July 1 to count for that year's season.  Members are encouraged to seek additional training (camps, seminars, conventions, etc.)  and anything attended on or after July 1 will count toward the following season. Any substitution to NBUA-provided training must be approved with the Director of Training prior to July 1 of that season.
\end{itemize}

High School Participation: To officiate High School games, umpires are required to take the \href{http://www.nfhs.org/}{National Federation of High Schools (NFHS)} clinic(s) provided by \href{http://wiaa.com/}{Washington Intercollegiate Athletic Association (WIAA)} / WOA, complete the NFHS test receiving a passing grade and any other WOA requirement annually. Each year prior to the start of the school ball season members will be notified when the clinics and tests are available as well as the deadline for completion. Failure to complete the NFHS clinics and test by the deadline will result in being ineligible to officiate school games for some or all of that season.

\newpage
\section{EVALUATIONS}
To assist umpires in improving their skills and to provide constructive feedback NBUA provides an evaluation program. The evaluation program is based on partner evaluations performed by on-field peers and formal evaluations done by paid evaluators.

\subsection*{Formal Evaluation:}
A formal evaluation involves a member of the development committee being assigned to evaluate one or more umpires for the entire game. In a formal evaluation the Developer will be in attendance for the entire game, will use a specific evaluation protocol to judge the umpire in a variety of categories and will assign a grade or value to each category. The Developer will also document specific observations both positive and areas for improvement. At the end of the game, the Developer will meet with the umpire(s) being evaluated and will review the evaluation. This evaluation will then be posted in the Arbiter website. 
The Developer will be paid \$40 for a formal evaluation of one umpire or \$60 for a formal evaluation of both umpires.  As it is difficult to focus on more than two umpires in a contest and maintain high quality, a Developer will be limited to evaluating two umpires in one game.
It should be noted in some cases the Developer may meet the umpire(s) to be evaluated prior to the game informing him/her of the evaluation however many times the Developer will attempt to make this formal evaluation a “blind evaluation” and will not inform the umpire being evaluated until the conclusion of the game.
\begin{itemize}
\item  The Director of Evaluations annually selects a committee, known as Developers, that are willing/able to perform formal evaluations as requested.
\item  Each year, the Director of Evaluations determines a target group of umpires for formal evaluation, and makes every attempt to manage those evaluations to completion through the evaluation staff.
\item  Due to limited resources, only a portion of members shall receive a formal evaluation each year.
\item  Members requesting a formal evaluation should notify the Director of Evaluations.  Requests will be prioritized according to available resources.
\item  On rare occasions, a second formal evaluation may be requested by a member.  However, this will be at the discretion of the Director of Evaluations. If a second formal evaluation is performed, the evaluation fee will be deducted from the member's game fee.
\end{itemize}

\subsection*{Partner Evaluation:}
A partner evaluation provides valuable information.  However, because the partner is also working it is not as thorough as a formal evaluation.  The partner evaluation won't be considered as strongly as a formal evaluation by the Evaluation Committee.  

The Apprentice umpires are expected to focus on officiating duties and are not required/expected to participate in giving evaluations, however, they are strongly encouraged to learn about the criteria by which they are evaluated, and to discuss with their mentor the evaluations they receive.

\subsection*{Evaluation Scoring}

\begin{tabular}{c l l}
\hline\hline
\multicolumn{2}{l}{\textbf{Formal Evaluations Scoring Criteria}} \\
\hline
1 & Consistently below minimum expectations \\
2 & Seldom meeting expectations \\ 
3 & Achieves all expectations \\
4 & Sometimes exceeds expectations \\
5 & Consistently exceeds expectations \\
\hline\hline
\end{tabular}

\subsection*{Evaluation Criteria}

In order to ensure consistency between evaluations, the same basic criteria will be used for formal and partner evaluations.

\begin{tabular}{c l l}
\multicolumn{2}{l}{\textbf{Plate Work}} \\
\hline
15\% & \textbf{Plate Positioning:} Proper position of head and body, stability, no flinching, \\ 
& style \& mechanics of balls, strikes, called foul tips \& foul balls. \\
15\% & \textbf{Judgment \& Timing of Calls:} Proper positioning for plays, point of plate, \\
& baseline extended, interference, dugout overthrows, fair/foul down the line, fly balls. \\
15\% & \textbf{Strike Zone:} Judgment, timing, and consistency. Always remembers the count. \\
& Uses a consistent tenor when calling strikes. Ball Calls are "dugout loud". \\
15\% & \textbf{Game Management:} Concise Plate Meeting, points of emphasis, ejections \& warnings, \\
& meetings, mound visits. Keeps the game moving, managing time-outs, and putting the \\
& ball back in play. \\
13\% & \textbf{Hustle \& Mobility:} Distance, proper footwork \& moves with a purpose. Shows \\
& effective mobility \& hustle throughout the entire game. \\
10\% & \textbf{Crew Mechanics:} Communication, positioning, rotations, signals-mirror, double-\\
& plays, run-downs, swipe-tags \& getting help on calls. Effectively engaged with \\
& Partner \& the game. \\
 9\% & \textbf{Appearance:} On time \& good pregame meeting. Proper, professional appear- \\
& ance \& demeanor. Minimizes fraternization with coaches, players and fans. \\
 8\% & \textbf{Rules Application(s):} Knows and properly applies rules, association policies/pro- \\
& cedures, league points of emphasis, and ground rules. \\
\hline
\end{tabular}


\begin{tabular}{c l l}
\multicolumn{2}{l}{\textbf{Base Work}} \\
\hline
18\% & \textbf{Hustle \& Mobility:} Getting distance, proper footwork \& moves with a purpose. \\
& Shows effective mobility \& hustle throughout the game. \\
15\% & \textbf{Position \& Rotation:} Proper positioning prior to pitch. Proper use of Pause, \\
& Read \& React. \\
15\% & \textbf{Footwork on the Bases:} Judgment and footwork on infield plays, positioning for \\
& outfield balls, infield flies, pick-offs, balks \& run-downs. \\
12\% & \textbf{From the Rail:} Judgment on when \& when not to go out, coming in to pivot, infield \\
& ground balls, fouls near the fence, fair/foul along first base line, swipe tags \& bunts. \\
12\% & \textbf{Base Calls Made:} Judgment, timing, and accuracy of calls. In proper position to see \\
& the play(s). \\
10\% & \textbf{Crew Mechanics:} Communication, positioning, rotations, signals-mirrored, double- \\
& plays, run-downs \& swipe-tags. Knows when to get help on calls. Effectively engaged \\
& with his partner \& the game. \\
10\% & \textbf{Appearance:} On time. Effective Pregame. Proper, professional appearance \& de- \\
& meanor. Minimizes fraternization with coaches, players and fans. \\
 8\% & \textbf{Rules:} Knows and properly applies Fed/OBR rules, association policies/procedures, \\
& league points of emphasis, and ground rules. \\
\hline
\end{tabular}

\subsection*{Tier level promotions}

At the end of each season the Evaluation Committee will meet to discuss and make recommendations for member tier promotions. These recommendations will be based on a variety of criteria including but not limited to formal evaluations, partner evaluations, and peer evaluations. These recommendations are forwarded to the Board of Directors and they will approve any and all tier level promotions.

To be considered for tier level promotion an umpire MUST: 
\begin{itemize}
\item  Work a minimum of thirty (30) games in the season.
\item  Attend the minimum required training provided prior to July 1.
\item  Have completed the NFHS clinics and passed the NFHS test.
\item  Promotion to Tournament and A tier requires a full evaluation that year
\end{itemize}

\subsection*{Tier level adjustment}
From time to time it may be determined that for whatever reason an umpire is not performing at the tier level where they are assigned. If this is observed a recommendation may be made to the Board of Directors to reassign that umpire to a lower tier level. Prior to reassignment, a formal evaluation must be conducted.

\newpage
\section{ASSESSMENTS AND FEES}
As NBUA is an association of members, the game fees earned belong to the officials that work the game.  NBUA assesses a portion of each game fee to cover association costs, as well as some annual and per-event fees.  All game fees earned by an official are remitted to that official minus the assessments.

A baseline of 8\% of game fees is assessed to cover NBUA operating costs.  The following annual and per-event fees are assessed:

\begin{tabular}{l}
\\
\hline\hline
\$65: NASO annual dues* \\
\$30: WOA dues, waived for board members \\
\$10: annual training fee for returning members \\
\$10: banquet fee for member, additional fees apply for guest(s) \\
Give-back charge, \$10 regular, \$20 if within 24 hours, per incident \\
\hline\hline
\\
* if not previously paid through another officiating association for another sport \\ 
\end{tabular}


\newpage
\section{UMPIRE INFORMATION}
The following are a series of areas important to each NBUA member. This section is meant to provide information on how to access information, how to comply with certain requirements and general information helpful to everyone. 

\subsection*{NBUA Website}
To facilitate communication and allow one Assignor to handle the number of games covered by NBUA we try to manage everything we can through the NBUA web site, through “Arbiter” and whenever possible via email. The NBUA website has a public section and two private sections, they are:
\begin{itemize}
\item  Public: Accessed through the \href{http://www.nbua.net/}{NBUA website} (www.nbua.net). This provides general information as well as a gateway to the private sections available only to association members
\item  Private: NBUA uses \href{http://www.ArbiterSports.com/}{Arbiter} (www.ArbiterSports.com) for setting availability, viewing game schedules, tracking of games worked, evaluations, membership rosters, earnings, and field locations.  The official's email address and password will also need to be supplied. Arbiter may also be accessed through the NBUA website by clicking on the Arbiter link then entering email address and password.  
\item  Private: Members Only: This section of the NBUA website provides association information that is restricted to the association membership only. Access to the private section requires a  password that will be provided during training. If assistance is needed with this password, email the Director of Member Services \& Communications. It is important to remember Arbiter is for the individual member and the Members Only section is for NBUA members only. Access to both sections is through a password and should not be given to others. 
\end{itemize}

\subsection*{Umpire Earnings}
In Arbiter there is a tab marked “Payments”. This tracks the game fees earned as well as any draws that may have been paid. 
\begin{itemize}
\item  Reporting of earnings: As an independent contractor of NBUA, all earnings will be reported to the IRS and an IRS Form 1099 will be provided. It should be noted, NBUA does not withhold any taxes and the member is responsible for all taxes as an independent contractor. 
\item  Requesting a draw: During the season periodic draws can be requested on earnings. To receive a draw an email must be sent to the Treasurer stating a request for a draw no later than the first day of the month.
\begin{itemize}
\item  Draw checks are normally processed on the 15th of the month and sent by mail.
\item  A draw of up to 70\% of earnings to that date may be requested.
\item  Draw checks cannot be requested until notified.
\item  While NBUA will always try to honor draw requests and honor the amount requested, all draws are subject to the Association having adequate cash on hand.
\item  In the event of an extreme hardship or emergency situation requests prior to June 1 will be considered by the Board of Directors on a case by case basis. 
\end{itemize}

\item  Extension of credit: NBUA will not extend credit to any association members and will not advance draws on game fees that have not been earned. 
\end{itemize}

\subsection*{Game fee schedule} 
The game fee schedule listing game fees for each level of game will be posted in the “Members Only” section of the NBUA website. There are a few special situations where mileage, ferry costs, parking costs is reimbursed (at a fixed rate). While that information is not on the game fee schedule, if an eligible game is worked it will be reflected on the earnings statement. 

\subsection*{Reports} 
There are several times when it is required that an umpire file a report. All of these reports can be filed online however in some circumstances we also require a phone call to the Assignor. 

\subsubsection*{Ejection Report} 
If an ejection occurs, the ejecting umpire must file an Ejection Report within 24 hours (prefer 12 hours) of the game. If this ejection occurred during a school game the Assignor must be contacted immediately following the contest providing basic information. Also, if the incident requiring an ejection is considered to be particularly egregious, in addition to filing a report, at a minimum, a phone call to the Assignor is required.
\begin{itemize}
\item  To file a report, go to the NBUA website (www.nbua.net), click the “Members” tab, then click on "Incidents". A password is required for the members section. Follow the instructions using the proper report.
\item  Complete the form with as much detail as possible. In completing an ejection report include actual words or language used.  Include any cuss/swear words as said with quotes around them.  Do not omit actual words within a quote.  Report facts and do not express opinions.
\item  Note: Ejections are taken seriously, ejection reports are forwarded to the appropriate Athletic Directors or League Presidents and often there are consequences for the actions. So be accurate, be honest, be complete, do not express opinions and provide all relevant information.
\end{itemize}

\subsubsection*{Injury report} 
If an injury occurs while umpiring a scheduled non-school game, please report the injury to the Assignor as soon as possible after the contest. If injured during a scheduled school contest, follow the \href{https://woa.arbitersports.com/front/104759/site/Officials/Insurance/}{procedure for submitting an L\&I insurance claim} via the WOA Central Hub.  For questions, please email Leah Francis at lfrancis@wiaa.com.

\subsection*{Customer complaint / feedback} 
If a customer wishes to file a complaint or provide any type of feedback, refer them to the \href{http://www.nbua.net/}{NBUA website} (www.nbua.net) and click on customer feedback. Any and all information from customers is welcome and appreciated.

\subsection*{Mentor program} 
To encourage new umpires and to provide them with assistance during their first two years with NBUA we established a mentor program. While we provide guidelines to both the mentor and mentee these are guidelines and the value to umpires will be proportional to the investment.
\begin{itemize}
\item  Mentor is asked to communicate regularly with mentee answering questions, giving encouragement and helping the new umpire in any way possible.
\item  We encourage the mentor and mentee to work a few games together early in the season to provide the mentee a good foundation for learning. Note, while the Assignor will attempt to match the mentor and mentee frequently, it is the responsibility of the umpires to communicate with the Assignor to coordinate schedules.
\item While the first year is normally the most critical for the mentee we ask a mentor to continue communication and assistance with each mentee through the second year.
\end{itemize}

\newpage
\section{GAME ASSIGNMENT RESPONSIBILITIES}
When a game has been assigned via Arbiter the umpire accepts responsibility for that game. The following briefly describes how games are assigned, notification of new games assigned, the umpire’s responsibility for those games, what to do in the case of an emergency and what to do in the case of inclement weather or unplayable field conditions.
\begin{itemize}
\item Availability: For games to be assigned the umpire must provide their availability via Arbiter
\begin{itemize}
\item  Blocking availability: To set availability go into account on the \href{http://www.ArbiterSports.com/}{Arbiter website} (www.arbitersports.com); go to the “Blocks” and view current availability. The key is to indicate the days and/or times within days that are NOT available to work. Typically, this would be the time spent at one’s primary job.
\item  If personal schedule varies blocks can be set differently for each day of the month.
\item  Selecting a date range to block all or partial days is also possible.
\item  Updating availability: If during the month availability changes for any given day that day must be edited or a game may be assigned.
\item  Travel limits and assignments: When entering availability it is possible to specify by day the travel limits from a selected zipcode from where the travel limit will be based. The Assignor will make every attempt to limit assignments to those geographical areas indicated, however, from time to time that may not be possible.
\end{itemize}
\item Game assignments: Games are assigned based on a set of criteria developed by the Board of Directors, the volume of games available for assigning and the availability posted by each member. 
\begin{itemize}
\item  Website assignments: All assignments are made through Arbiter. It is the responsibility of each association member to monitor their schedule for assigned games.
\item  Email, text messages on assignments: If the member has provided information in the “Profile” section of the website, when a new assignment is made a message indicating “You have new assignments for NBUA” with link to Arbiter will be sent. Once Arbiter is opened, responsibility for the game is accepted and the umpire is now responsible for the game. 
\end{itemize}
\item Assignment changes: From time to time after a game has been assigned it may be necessary for the Assignor to make changes to an assignment. When that happens, the change will occur on Arbiter and the umpire will receive an email or text message indicating the new assignment. If a change needs to be made within 24 hours of your scheduled game, the affected officials will be contacted by phone, email, or text message. 
\item Canceled games: Occasionally a team or league will cancel a game. When that happens, the canceled game will be noted on Arbiter and the umpire will receive an email or text message indicating the game has been canceled.  
\item Rain cancellations: Since we live and umpire in the Pacific Northwest, rain outs are a fact of life. The process for rain or inclement weather cancellations is as follows:
\begin{itemize}
\item  When the Assignor receives the cancellation, the change is made to Arbiter and an email or text message goes out to the umpire indicating their game has been canceled.
\item  During the season the NBUA “status phone line” is updated as often as possible.
\item  If a game has not been canceled on Arbiter and/or is not listed on the status phone line despite weather conditions the umpire is required to report to the field.
\item  If a game is canceled less than 2 hours before game time or if the game is canceled at the field by the coach or athletic director the umpire will receive full game fee. If a game is canceled at the field prior to the start of the game by an umpire, the umpire will receive a full game fee.
\item  Once a game has started stopping the game due to rain or inclement weather is the responsibility of the umpires. If the umpires determine the field condition presents a danger to the players, they must stop the game. The umpires will then wait for 30 minutes to determine if the game can be continued before leaving the field. Note: If it is obvious the field will not become playable and both coaches agree, the 30-minute waiting period can be waived.
\end{itemize}
Finding a replacement umpire: By carefully managing and updating availability, the need to find a replacement umpire should be minimal, but occasionally that happens. If a situation arises where a replacement is needed for an assigned game the Assignor must be contacted. The Assignor is the only person who can reassign games for NBUA. Note: When asking for a game to be reassigned the Assignor MUST be given adequate time to find a replacement umpire.
\item A fee of \$10 for every give-back game will be charged, but there are some exceptions: 
\begin{itemize}
\item  Each umpire may use one "free" give-back each year
\item  If the Assignor initiates a change in the schedule, there is no charge for a give-back
\item  If an assignment conflicts with posted availability, and the Assignor is notified immediately, there will not be a charge for a give-back
\end{itemize}
\item In addition to the \$10 give-back fee, a penalty of \$20 will be assessed to any umpire who gives a game back within 24 hours of the start time of the game. If there are extenuating circumstances, this penalty may be appealed to the Member Liaison. If the appeal has merit, the Member Liaison shall forward his/her recommendation to the Board of Directors for a final determination.
\item If the request is within 24 hours of the start of the game, the Assignor must be called in addition to email. If no answer, leave a message. The game will be reassigned. 
\end{itemize}
Give-backs are tracked by the Assignor and the Director of Evaluations. Umpires that give back an inordinate number of games may no longer receive assignments.

\subsection*{Assigning Process Policy}
Because NBUA represents umpires across a broad spectrum of skills and experience levels, and supports customers in many age groups and levels of baseball, it can be confusing to individual umpires about why certain games are assigned and others are not.  Umpires typically have visibility to their own game assignments and only periodically get glimpses into the assignments of other umpires, so these assignments can often seem unequal or unfair.  Assigning games is a complex and nuanced process that requires years of experience.  In order to clarify the process, game assignments are based on the following general guidelines: 

\begin{tabular}{cccc}
Tier & School Level & Summer Level & Exceptions \\
\hline\hline
TT & Varsity & All & \\
A & Varsity & 18U & PIL with TT \\
B & JV & 16U/Adult & 18U/Varsity With A or above \\
C/1st Yr & JV/Fresh/Jr High & 14U & 16U/Adult With B or above \\
\hline\hline
\end{tabular}

\begin{itemize}
\item Umpires specify availability in Arbiter, consisting of days and times they are not available for assignment.
\item Arbiter supports travel distance limits outside of which games should not be assigned.
\item Tiers are guidelines for assigning and should not be construed as ensuring umpires exclusive rights to any game at any level.  Top umpires work lower level games with less experienced umpires for training, working with a variety of other umpires, and in cases of need and availability.
\item The assignor does not give exclusive assignments from top tier to bottom.  Games are spread out across qualified umpires over tiers to ensure a mixture of umpires are working on a daily basis.
\item Assignor attempts to assign games primarily within tier; however, all umpires should be willing to work all levels of baseball.
\item Highly competitive summer tournaments (Mack/Elite 18) are typically covered by TT and top end A tier umpires.
\item Competitive summer tournaments (Mantle/Elite 16) are covered by TT, A, and accomplished B tier umpires.
\item As lower tier umpires demonstrate acumen to handle higher level games, opportunities arise to assign more competitive games with more experienced partners in mentor situations.
\item When assignor is in need of assigning multiple games to umpires in a day, effort will be made to limit the number to two games.  If/when the need arises to schedule more than two games in a day, effort will be made, consideration is made to whether the umpire is physically capable and willing to accept the games.  Scheduling more than 2 games per day will be done rarely so as to avoid umpire fatigue.  If/when more than three games need to be assigned to an umpire in a day, games of a less competitive nature will be included when possible.
\item When at all possible, umpires will not be assigned back-to-back days with more than two games so as to avoid umpire fatigue.
\item When assignor is unable to cover games after exhausting all internal resources, neighboring associations will be contacted.  If games are still uncovered after enlisting help from neighboring associations, fully covering games received first is higher priority.  Late additions, regardless of their level, are given lower priority to encourage and support customers that plan and coordinated well.  Customers with last minute demands will be encouraged to move games to less busy days.
\item The assignor encourages customers to have game schedules submitted at least two weeks in advance so that umpires can be assigned to games one to two weeks in advance throughout the season.  One tool that is used is to apply a late schedule fee to games scheduled less than one week in advance where no previous communication was received, with the exception of rain out or other rescheduled make up games.
\item When the association is at capacity for covering existing games, additional customers may also be turned away rather than to over assign umpires and risk abusing the principles described above.
\item While NBUA is always interested in finding new customers, supporting existing customers well is considered first.  The association resists the urge to drop lower levels of baseball and younger age groups to ensure that NBUA offers a balanced mixture of games for umpires to work and gain experience across all tiers.
\end{itemize}

\newpage
\section{GAME DAY}
On the day of a game it is vital the umpiring crew is prepared to provide professional umpiring.

\subsection*{Arriving at the field}

\begin{itemize}
\item Both umpires should arrive at a minimum of 45 minutes before game time.
\item Umpires should communicate with each other to make sure they know where one another will be parking.
\item Umpires should attempt to park away from the field, away from the parents and players and whenever possible in an area where they are not in full view of the public.
\item As discussed in the “Expectations” section, umpires must use decorum and common sense when changing into their uniforms in the parking lot.
\end{itemize}

\subsection*{Entering the field of play}
\begin{itemize}
\item The umpiring crew must enter the field 10 minutes prior to game time.
\item All members of the umpiring crew must enter the field at the same time. 
\end{itemize}

\subsection*{Game Management}
The umpiring crew is responsible for professional management of all games and as such are expected to act in a professional manner at all times. To set the tone for the game and to manage the game professionally, among other things the umpire crew is expected to:
\begin{itemize}
\item  Arrive on time to the field of play, enter as an umpiring crew and immediately proceed to the plate to conduct the plate meeting
\item  Conduct a professional plate meeting as described during training.
\item  Manage all conflicts and disagreements in a professional manner:
\item  Do not engage in a verbal exchange across the field with a coach.
\item  When engaging in a conversation with a head coach, always act professionally, do not shout back and do not use inappropriate language despite what the coach may be doing.
\item  Handle all disputes and confrontations as discussed during training.
\item  If an ejection occurs, it is the responsibility of the non-ejecting umpire to escort the ejected player/coach from the field.
\item  If fan activity is inappropriate and distracting from the game, communicate with the head coach(s) and ask them to rectify the situation. Umpires should not interact directly with fans. If the coach is unable or unwilling to rectify the situation there is the option to eject the coach and in extraordinary circumstances terminate the game.
\end{itemize}

\subsection*{Suspension of a game (weather / darkness)}
Once the game has started, provided there are safe playing conditions, management of the game is the responsibility of the umpiring crew. While everyone wants to complete the game, safety of the participants must take priority and is the responsibility of the umpires. After a game has started a coach, athletic director, tournament director or league official cannot overrule the umpiring crew in determining if playing conditions are safe or if a game will be temporarily or permanently suspended. 
\begin{itemize}
\item  Rain: In the event of rain during the game the following guidelines should be followed:
\begin{itemize}
\item  If in the judgment of the umpires the rain presents a dangerous condition the game shall be stopped.
\item  If in the judgment of the umpires the field becomes slippery to the point where a player could be injured due to the field conditions the game shall be stopped. 
\item If the contest is being played on an artificial turf field, coaches sometimes feel that means the game can go the entire distance of the game even with inclement weather.  If in the judgment of the umpires, even with artificial turf, the field is unsafe, unable to keep balls dry or a player could be injured, the game shall be stopped.
\item  Lightning: Lightning presents an immediate and extremely serious danger to players, coaches, fans and umpires. In the event of lightning the following must occur:
\item  If lightning is observed anywhere in the sky they game must be immediately stopped, and all players must leave the field. While at that point the players become the responsibility of the coaches they should be encouraged to find shelter somewhere other than in the dugout if the dugout is constructed with chain link fencing. 
\item  If thunder is heard but lightning is not seen, and it appears the thunder is close to the field the same procedure should be followed.
\end{itemize}
\end{itemize}

\subsection*{Resuming the game}
\begin{itemize}
\item Games should be resumed and completed, when possible, following these guidelines:
\begin{itemize}
\item  Rain: If a game is stopped due to rain the umpires are required to wait 30 minutes to see if the rain stops and the field can be made playable. If during the rain delay it is obvious the rain is not going to stop and both coaches agree continuing the game will not be possible the game may be suspended before 30 minutes have elapsed. 
\item  Lightning: If a game has been suspended due to lightning the game cannot resume unless there has been no observance of additional lightning for a 30-minute period. Again, lightning can be extremely dangerous thus extreme caution must be exercised before resuming a game.
\item  Darkness: If the umpire crew determines it is no longer safe to continue a game due to darkness it is their responsibility to suspend play. 
\item  When possible attempt to stop the game at the end of an inning.
\item  If it becomes obvious it will not be possible to complete a game but could complete one more inning inform the coaches that this will be the last inning.
\item  If a “last inning” drags on to the point where it becomes dangerous to play, play must be suspended. 
\item  Loss of lights: Most fields that have lights either have a “lights out curfew” or the lights are on a timer. When officiating games on fields with lights the following must occur:
\item  At the plate meeting, lights out curfew, timer issues, and time allowance for safe exit from the field, must be discussed.
\item  In either case umpires must immediately stop the game ten (10) minutes prior to "lights out curfew" regardless of where the game is.
\item  If the game is close to the lights out time at the end of an inning and it is obvious a complete new inning cannot be completed the game should be stopped at that time. 
\end{itemize}
\end{itemize}

\subsection*{After the game}
\begin{itemize}
\item After the game the umpire crew shall leave the field together and return to their cars together.
\item If fans, coaches or players approach umpires in the parking lot to discuss or argue aspects of the game, the umpires should not engage in those discussions, but politely say the game is over and they are not discussing it further.
\item Post-game discussions between the umpiring crew should involve discussions about any unique plays that may have occurred and providing feedback to each other in a way that will assist them in the continued improvement of their umpiring skills.
\item Umpires shall leave the parking lot at the same time.
\item If threatened, umpires should call 911, warn the offender(s) that the authorities have been contacted, and contact the assignor.
\end{itemize}

\newpage
\section{POST SEASON AND SPECIAL EVENTS}

NBUA provides umpires for a variety of special situations including post season high school games, summer tournaments and special events such as charity and other games. 

\begin{itemize}
\item \textbf{High School post season assignments:} To be eligible to be considered for a post season assignment, the umpire must be a member in good standing, must be a permanent A or Tournament tier umpire, must be WOA RTO (Retention, Training, and Observation of officials program) certified and must have met the WOA minimum number of High School Varsity games (number determined by Board of Directors). In order to set expectations, the number of assigned games will be communicated prior to the season.  Additionally, the umpire must be available for assignments at any location in the State of Washington and must be available for all potential assignment dates. 

Based on the number of post season slots given to NBUA by the WOA, the Board of Directors will compile a list of umpires eligible for post season assignment and will request availability from those umpires. This list will be based on the criteria listed above and will attempt to rotate post season assignments to give all of our qualified umpires an equal opportunity. The Board of Directors will then submit that list to the WOA and it is WOA who makes the assignments including the geographical location with NBUA having no further input and no control of location or game level assignments.  WOA will make assignments via Arbiter.

\item \textbf{Special Event Assignments:} From time to time we are given the opportunity to provide umpires for special events including charity and other games often played at premier fields. When those opportunities present themselves, it is the responsibility of the Assignor working with the Board of Directors to make those assignments. While an effort will be made to provide these opportunities to as many umpires as possible most of these assignments require higher level umpires. It should be noted that some of these assignments are not compensated.
\end{itemize}

\newpage
\section{IMPROPER CONDUCT}

All Association members have the right and responsibility to report improper conduct of any Association member.  Association members have the right to be notified in writing concerning any reports (as set forth in Article 8.5a) of their alleged improper conduct, including the date, location, nature of the alleged improper conduct and the reporting member(s).  In all matters involving alleged improper conduct, the member charged with improper conduct shall be accorded due process.  Misconducts can include but not limited to those outlined in Article IX, section 6, of the WOA Bylaws such as Using abusive language or distasteful gestures, using substances the day of the game that could impair judgment (i.e. alcohol, marijuana, or even certain prescription drugs).

\begin{itemize}
\item Reports

Reports alleging improper conduct whether from inside or outside of the association, must be documented and submitted in writing (such as by telephone memo if not initially received in writing) and sent to any member of the Board of Directors. Upon receipt, the Board shall appoint an Ethics and Grievance Committee Chairman, who shall select two non-board members in good standing to form his committee.  Reports shall include, to the extent available, the name of the member(s) and other person(s) involved, the date, location, nature of the incident, and the name of the complainant.  Reports of improper conduct shall be handled discreetly and professionally by the Board, the Ethics and Grievance Committee, and by members with knowledge of the reported conduct.  The Ethics and Grievance Committee Chairman and his committee will conduct a review of the report to determine action.  If it is deemed a grievance, the committee shall follow the procedures outlined below.  The Chairman will then make a report to the Board at the conclusion of the grievance procedure.  In the case of alleged misconduct, the committee shall appoint an investigator within five (5) business days.  The Board shall have the option to provide a list of investigators (non-board members) suitable for the cause.  The member shall be contacted to inform him/her of the allegation at the same time the matter is turned over to an investigator.

\item Investigation

The investigator shall use discretion and take reasonable steps to maintain confidentiality during and after the investigation. The investigator shall interview the accused member(s), involved parties, and identifiable witnesses to the allegation. The investigator shall render a report of the facts discovered during the investigation and an assessment regarding the accuracy of the allegation and deliver the report to the Ethics and Grievance Chairman within seven (7) business days.  During the period of the investigation, the accused member may be temporarily suspended by the association. In such case, the accused member must be notified of such action.

\item Judgment

Upon receipt of the report from the investigator, the committee shall make a judgment within forty-eight (48) hours.  If the allegation is found to be unsubstantiated, the committee will forward the recommendation for immediate dismissal to the board and recommending the member to immediately be reinstated as a member “in good standing” and any temporary suspensions, withdrawn immediately.  If the allegation is upheld, the committee shall recommend sanction(s) to be imposed to the Board within five (5) business days. The Board shall approve or modify the recommended sanctions from the committee and notify the member. The decision shall be set forth in a Letter of Findings. The Letter of Findings will be signed by the President. The Letter of Findings shall include the investigator report, the findings and judgment of the committee, any sanctions imposed by the Board, and available options for appeal.  The Letter of Findings shall be sent to the member(s) via certified mail, return receipt, postage prepaid, to the mailing address of the member(s) then on file with the Association, and to the member’s email then on file.

\item Sanctions

Sanctions may include, but are not limited to probation, suspension, or expulsion.

\item Appeal

The decision may be appealed by the accused member(s) by delivering, in writing, via E-mail, in person, or by certified mail, return receipt requested, postage prepaid to the President of the Association, a request for an appeal hearing within seven (7) calendar days of receipt of the Letter of Findings or, in the absence of documented receipt of the Letter, within seven (7) calendar days of the postmark of the Letter of Findings. The Board shall hear the appeal at the earliest convenience and shall make a decision on the matter and render a final written determination within fourteen (14) calendar days from the date of the appeal hearing.  An audio recording or verbatim record shall be made at the proceedings of the appeal (the record to be retained by the Secretary).

The Board may, by majority vote:
\begin{itemize}
\item Uphold the appeal by the member and reverse the prior decision
\item Confirm the original decision
\item Confirm the original decision, but modify the specified sanctions
\end{itemize}

Failure by the Board to achieve a majority to uphold the appeal will constitute acceptance of the findings including sanctions and/or corrective action. The Board’s decision shall be final for NBUA and shall be communicated to the member(s) involved, in writing, via email and delivered via certified U.S. Mail, return receipt requested, postage prepaid, to the member’s mailing address then on file with the Association, and to the member’s email then on file..

Failure to request an appeal hearing in the manner herein described and within the time frame set forth shall be deemed a waiver of the right to an appeal hearing under these procedures including for any determinations of corrective action, disciplinary action, suspension, expulsion or other action by the Association. Any subsequent decision to reinstate membership privileges will be determined by a majority
vote of the Board of Directors. 

In the event, that the member is not satisfied with the results of the appeal to the Board of directors and the sanction(s) in any way impacts eligibility for games or other matters under the jurisdiction of the WOA, the member has the right to appeal to the WOA.  In any case, he/she must be informed of this right at the conclusion of the findings of the Board of Directors.

\newpage
\item Grievance Procedure 

\begin{itemize}
\item Procedure – following paragraphs outline potential procedure for a member “in good standing” to initiate a grievance against policy and procedures.
\item Purpose – the purpose of this provision is to prescribe procedures if an official wishes to initiate a grievance against the association
\item Definition – a grievant shall mean an active member official “in good standing” making an allegation of a violation, misinterpretation, or misapplication of the NBUA’s policy and procedures.
\item Limitations – All formal grievances shall be initiated by a grievant within five (5) days of the date such grievance is discovered or reasonably should have been discovered.  A grievance not presented in accordance with the foregoing shall be considered waived by the grievant and will be denied.  The grievance procedures herein shall be the method by which grievances are resolved.
\item Informal Procedures – A grievant shall attempt to resolve the situation by meeting with the Ethics and Grievance Committee Chairman.  Two (2) days shall be allowed for this process and possible resolution.  Grievances involving more than one member, shall be dealt with as one grievance, unless separate and specified allegation are evidenced by the grievance.  
\item Formal Procedure – If the grievance cannot be resolved with the informal procedure, then within the time period of two (2) days, the grievance shall present in writing specified allegation to the Ethics and Grievance Committee Chairman.  The date for a hearing will be set within five (5) days after receipt of the written allegations.  The Ethics and Grievance Committee shall hear the grievance.  A decision on the grievances shall be within two (2) days.  A written decision to the grievant shall be forthcoming within three (3) days following the hearing.
\item Appeal - If the decision of the Ethics and Grievance Committee is not accepted, or the decision fails to meet the above deadlines, or the grievance procedures have not been adhered to, the grievant shall have five (5) days to file an appeal.
\end{itemize}

\end{itemize}

\newpage
\section{Consequences}

NBUA provides opportunities for game assignments appropriate to the skill level of the member and the needs of the association. NBUA advocates for its members and defends their due process in the face of complaints or accusations from other members, customers, or others. However, when behavior of a member was deemed not consistent with the goals, quality, image, or standards of NBUA, or that there were violations of NBUA’s bylaws or policies, and procedures, disciplinary action may be required.

A discipline policy is necessary for a healthy organization where officials can grow and thrive. Standard disciplinary options create fairness, consistency, predictability, and accountability over time and across a wide set of situations.

NBUA resists calls by customers to refuse individual officials by name and discourages officials from refusing to work with other members by name.

The desired outcome of the disciplinary process is that the member corrects the behavior and returns to good standing.  The consequence of not correcting the behavior could be subsequent disciplinary action. 

No official that is under disciplinary action shall be eligible for post-season school contests, subject to policies regarding opportunities for appeal.

\subsection{Probation}

Typically, a first offense will call for communication entitled Notice of Probation to be sent from the Board of Directors to the individual.  The letter shall describe the specific details regarding how the member has deviated from association policy or standards and the probationary time period.  While under probation, the member shall be subject to close scrutiny as deemed appropriate by the Board.  Scrutiny may include a formal or partner evaluation to assess the situation, at the member’s expense, deducted from game fees.  At the end of this probation period, at the determination of the Board, the probation may be dismissed, or additional discipline may be assigned by the Board.

\subsection{Suspension}

For more serious or repeated violations, the Board of Directors may determine that suspension is appropriate.  The member shall be informed by letter entitled Letter of Suspension.  The letter should describe the infraction(s) and time period of the suspension.  During the suspension, the member shall not be assigned games.  At the discretion of the Board, a probationary period may follow a suspension.

\subsection{Expulsion}

For the most serious violations, the Board of Directors may expel the official, terminating the member’s membership and standing with NBUA.  The member shall be informed by letter entitled Letter of Expulsion, describing the infraction(s).

\subsection{Resolution}

Upon favorable resolution of disciplinary action of any kind, the Board of Directors will notify the member that their standing is restored.  This shall be titled a Notice of Return to Good Standing.  The matter should be considered closed with no additional or remaining sanctions.

\section{Social Media Guidelines}

NBUA respects the right of all members to express their views publicly. At the same time, we must be aware of the pitfalls of social media.  Remember that the responsibilities of NBUA membership don’t end at the dugout or parking lot.

In public forums, it is difficult to clarify or retract statements. Comments can be taken out of context, humor misunderstood, feelings can be hurt, and personal and business relationships can be damaged. These guidelines are to assist NBUA members in participating online in a manner that protects the mission and reputation of NBUA.

Social media is defined as forms of electronic communication (such as websites for social networking and microblogging) through which users create online communities to share information, ideas, personal messages, and other content.

\subsection{Guidelines}

\begin{itemize}
    \item NBUA affirms and supports the \href{https://www1.arbitersports.com/Groups/104759/Library/files/SocialMediaGuidelines.pdf}{Social Media Guidelines} of the WOA.  Read these guidelines and become familiar with the NBUA By-Laws and Statements of Policies and Procedures.
    \item Personal social media posts and comments must not potentially embarrass NBUA, damage the NBUA reputation, or tarnish its image.  
    \item Comments must not damage the reputation or tarnish the image of NBUA partners or clients, including other officials, assignors, coaches, players, parents, administrators, or field personnel.  Avoid identifying people by name.
    \item Do not post false or misleading content.  Be genuine and do not attempt to misrepresent identities online.  Only the Member Services and Communications Director of NBUA (or designate), is authorized by NBUA to make official public statements representing the association.
    \item Members shall not create NBUA specific social media profiles or create websites, pages, or blogs that represent themselves as official NBUA content, without the written consent of the Member Services and Communication Director of NBUA.
    \item These guidelines are not meant to apply to various private forms of communication between parties, electronic or otherwise.  However, remember that anything written can be made public. Even benign comments can be misconstrued. 
    \item Do not post information about specific games, locations or dates that is not public domain. Posting scores, outcomes, situations, rulings and other specific information may not be consistent with the social media guidelines of our partners and is not in the scope of an NBUA member’s authority. General rule: when in doubt, don’t post it.
    \item Online relationships with coaches, athletic directors, players, or any other participants in games that NBUA members officiate should be carefully managed on social media. Keep potential changes in the relationship in mind. For example, a sports official may become a coach, player, or parent of a player. Care should be used to avoid perception of bias due to a public online relationship.
\end{itemize}

\newpage
\section{Member Status}

An individual's status with the association determines the rights, privileges, expectations, and financial obligations between NBUA and the individual.

\begin{itemize}

\item Member In Good Standing (MGS)

A regular member, dues/fees paid, WOA high school certified, training requirements met for tier level.  A member in good standing may be under “probation” meaning that for disciplinary reasons, additional scrutiny is applied to protect the reputation of the association.

\item MGS – Leave of Absence (LOA)

An otherwise MGS but is on “leave of absence” (LOA).  Any MGS can formally request time off which will be granted, but the member must then formally request to return to normal status.  Return to normal status is at the discretion of the board, who may consider duration away and other factors in determining tier level or may reject the return request and change status to Former MGS. When a training/dues/fees/certification deadline is reached and requirements have not been met, the member status changes to Former MGS.

\item MGS – Suspended 

Due to disciplinary action, the member is suspended.  When returning from suspension (see policies for conduct) to regular status (terms of suspension met), the member’s prior tier, training, certification, and other requirements resume where they left off.  This carries forward to the next year until a training/dues/fees/certification deadline is reached and requirements not met.  At that time, the member status changes to “Former MGS”.  Not invited to banquet/golf tournament and other association functions while under suspension.  Not eligible for transfer on letter from association until all sanctions fulfilled.

\item Expelled

Due to disciplinary action, the member has been expelled from the association.  NBUA will refund fees if required by WOA.  If an expelled official petitions to return to NBUA in the future, and that request is granted, they are subject to training, certification exams or other requirements and former tier level subject to re-evaluation and not guaranteed.

\item Former MGS

Has been a MGS previously, but is not a MGS for current year.  Includes members that have transferred to another WOA or non-WOA sanctioned umpire association.  If the former member wishes to attend the banquet or other functions and pays applicable fees, they are welcome and may request an invitation from Member Services chairman.  If a former member wishes to returns to NBUA as MGS in the future, they are subject to training, certification exams or other requirements and former tier level subject to re-evaluation and not guaranteed.

\item Guest

At times, NBUA allows non-members to assist in filling games.  These officials must be members of an accredited baseball association of umpires and not under any kind of disciplinary action.  Game assignments for any games appropriate to member C tier officials.  This is so that we do not take away games from members.  If no members are available for the open games and training and evaluation directors have no objections, then exceptions can be made on a case by case basis.  If concerns arise, non-member could be subject to training, certification exams, evaluations, or other requirements as specified by training or evaluation director, and continued assignments not guaranteed regardless of prior assignments.

\item Transfer

A member of another WOA (or other) accredited baseball officials association that is requesting to transfer to NBUA.  Policies already state that this must include a reference from the home association President or Assignor.  WOA specifies NBUA is obligated to automatically accept transfers from other WOA associations, although NBUA may appeal the transfer.  Game assignments for any games appropriate to member C tier officials.  This is so that we do not take away games from members.  If no members are available for the open games and training and evaluation directors have no objections, then exceptions can be made on a case by case basis.

\end{itemize}

\begin{tabular}{lcccccccc}
\\
\hline\hline
 &  &  & MGS &  & For- &  &  &  \\ 
 &  & MGS & Susp- & Exp-  & mer &  &  & Appr- \\
 & MGS & LOA & ended & elled & MGS & Guest & Xfer & entice \\ \hline
Receives Association &  \checkmark &  \checkmark &  \checkmark &  &  &  &  &  \checkmark \\
Notifications & & & & & & & & \\ \hline
Listed In Arbiter &  \checkmark &  \checkmark &  \checkmark &  &  &  \checkmark &  \checkmark &  \checkmark \\
Roster & & & & & & & & \\ \hline
The Year "Counts" &  \checkmark &  \checkmark &  \checkmark &  \checkmark &  &  &  &  \checkmark \\
for Seniority & & & & & & & & \\ \hline
WOA Membership &  \checkmark &  \checkmark &  \checkmark &   &   &   &   &  \checkmark \\
Included/Expected** & & & & & & & & \\ \hline
Can Be Assigned Games &  \checkmark &  \checkmark &   &   &   &  \checkmark &  \checkmark &  \checkmark \\ \hline
Normal Pay Schedule &  \checkmark &  \checkmark &  \checkmark &   &   &   &   &  \checkmark \\ \hline
Pay Schedule Exception &   & \checkmark &  &  \checkmark &   &  \checkmark &   &  \\ \hline
Invited to Association &  \checkmark &  \checkmark &  &  &  \checkmark &   &   &  \checkmark \\
Meetings/Functions & & & & & & & & \\ \hline
High School Eligible* &  \checkmark &   &   &   &   &   &   &   \\ \hline
High School Post- &  \checkmark &   &   &   &   &   &   &   \\
Season Eligible* & & & & & & & & \\
\hline\hline \\
\end{tabular}

* Additional tier/training/certification/other requirements may apply

** Any non-WOA member must pass background check for youth assignments

\end{document}
